\documentclass[twocolumn]{scrartcl}
\usepackage[sexy]{evan}
\renewcommand\ii[2][]{\item[#1] \textsl{\pinyin{#2}}}

% Definitions of Pinyin
\makeatletter
\def\py@yunpriv#1{%
  \if a#1 10\else
  \if o#1 9\else
  \if e#1 8\else
  \if i#1 7\else
  \if u#1 6\else
  \if v#1 5\else
  \if A#1 4\else
  \if O#1 3\else
  \if E#1 2\fi\fi\fi\fi\fi\fi\fi\fi\fi0
}
\def\py@init{%
  \edef\py@befirst{}%
  \edef\py@char{}\edef\py@tuneletter{}%
  \def\py@last{}%
  \def\py@tune{5}%
}
% Usage:
% \pinyin{Hao3hao3\ xue2xi2} (好好学习)
\def\pinyin#1{%
  \edef\py@postscan{#1}%
  \py@init
  % scan
  \loop
  \edef\py@char{\expandafter\@car\py@postscan\@nil}%
  \edef\py@postscan{\expandafter\@cdr\py@postscan\@nil}%
  \ifnum 0 < 0\py@char
    \edef\py@tune{\py@char}%
    \py@first \py@tuneat\py@tuneletter\py@tune \py@last\kern -4sp\kern 4sp{}\py@init
  \else
    \ifnum\py@yunpriv\py@char > \py@yunpriv\py@tuneletter
      \edef\py@tuneletter{\py@char}\edef\py@first{\py@befirst}\def\py@last{}%
    \else
      \edef\py@last{\py@last\if v\py@char\"u\else\py@char\fi}%
    \fi
    \edef\py@befirst{\py@befirst\if v\py@char\"u\else\py@char\fi}%
  \fi
  \ifx\py@postscan\@empty\else
  \repeat
}
\let\py@macron \= %chktex 14
\let\py@acute \'  %chktex 14
\let\py@hacek \v  %chktex 14
\let\py@grave \`  %chktex 14

%% \py@tuneat{Letter}{tune}
\def\py@tuneat#1#2{%
  \if v#1%
    \py@tune@v #2%
  \else
  \if i#1%
    \py@tune@i #2%
  \else
    \ifcase#2%
      \or\py@macron #1\or\py@acute #1\or\py@hacek #1\or\py@grave #1\else #1%
    \fi
  \fi\fi
}
\def\py@tune@v#1{{%
    \dimen@ii 1ex%
    \fontdimen5\font 1.1ex%
    \rlap{\"u}%
    \fontdimen5\font .6ex%
    \ifcase#1%
      \or\py@macron u\or\py@acute u\or\py@hacek u\or\py@grave u\else u%
    \fi
    \fontdimen5\font\dimen@ii
  }}
\def\py@tune@i#1{%
  \ifcase#1
    \or\py@macron \i\or\py@acute \i\or\py@hacek \i\or\py@grave \i\else i%
  \fi
}
\makeatletter
% End of pinyin

\begin{document}
\title{Chinese Terminology Sheet}
\subtitle{Simplified characters version}
\date{30 October 2022}

\twocolumn[
  \begin{@twocolumnfalse}
  \maketitle
  \begin{abstract}
  Chinese terminology sheet initially compiled as part of preparation
  for the Taiwan IMO 2014 camps, but later expanded.
  There are likely many typos.
  Please email any corrections to \texttt{evan@evanchen.cc}.
  \end{abstract}
  \vspace*{1ex}
  \end{@twocolumnfalse}
]


\section{General Math}
\begin{description}
  \ii[值]{zhi2} value
  \ii[取]{qu3} select
  \ii[由]{you2} from
  \ii[此]{ci3} this
  \ii[令]{ling4} let
  \ii[已知]{yi3zhi1} known
  \ii[考虑]{kao3lv4} consider
  \ii[分别]{fen1bie2} respectively
  \ii[是否]{shi4fou3} whether
  \ii[确定]{que4ding4} determine
  \ii[判断]{pan4duan4} determine
  \ii[达]{da2} achieve
  \ii[从未]{cong2wei2} never
  \ii[定义]{ding4yi4} define
  \ii[欲]{yu4} we wish to
  \ii[今]{jin1} now
  \ii[某]{mou3} some, certain
  \ii[若干]{ruo4gan1} some
  \ii[的数量]{de5shu4liang4} the number of \dots
  \ii[的个数]{de5ge4shu4} the number of \dots
  \ii[按上述]{an4shang4shu4} from above
  \ii[末]{mo4} last
  \ii[性质]{xing4zhi4} property
  \ii[端]{duan1} end
\end{description}

\subsection{Logical Conjunctions}
\begin{description}
  \ii[若]{ruo4} if
  \ii[则]{ze2} then
  \ii[故]{gu4} therefore
  \ii[因此]{yin1ci3} therefore
  \ii[即]{ji2} i.e.
  \ii[且]{qie3} and
  \ii[及]{ji2} and
  \ii[亦]{yi4} also
  \ii[根据]{gen1ju4} according to
  \ii[使得]{shi3de2} such that
  \ii[具有]{ju4you3} posses (a property)
  \ii[满足]{man3zu2} satisfy
  \ii[符合]{fu2he2} satisfy
  \ii[此处]{ci3chu4} here
  \ii[其中]{qi2zhong1} \dots where \dots
  \ii[与]{yu3} with
\end{description}

\subsection{More Logic}
\begin{description}
  \ii[任意]{ren4yi4} any
  \ii[皆]{jie1} all
  \ii[均]{jun1} all
  \ii[对于任何]{dui4yu2ren4he2} \dots for any
  \ii[对所有的]{dui4suo2you3de5} \dots for all
  \ii[某些]{mou3xie1} some, certain
  \ii[存在]{cun2zai4} exists
  \ii[恰]{qia4} exactly
  \ii[唯一]{wei2yi1} unique
  \ii[必]{bi4} must
\end{description}

\subsection{Technical Terms}
\begin{description}
  \ii[严格]{yan2ge2} strictly
  \ii[相异]{xiang1yi4} distinct
  \ii[两两相异]{liang3liang3xiang1yi4} pairwise distinct
  \ii[有限]{you3xian4} finite / bounded
  \ii[无穷多]{wu2qiong2duo1} infinitely many
  \ii[当且仅当]{dang1qie3jin3dang1} if and only if
  \ii[若且唯若]{ruo4qie3wei2ruo4} if and only if
  %\ii[充分必要条件]{chong1fen4bi4yao4tiao2jian4} necessary and sufficient condition
  \ii[充要条件]{chong1yao4tiao2jian4} necessary and sufficient condition
  \ii[类似地]{lei4si4de5} similarly
  \ii[互]{hu4} mutual
  \ii[彼]{bi3} that, those, one another
  \ii[固定]{gu4ding4} fixed
  \ii[等价于]{deng3jia4yu2} is equivalent to
  \ii[方程]{fang1cheng2} equation
  \ii[公式]{gong1shi4} formula
  \ii[陈述]{chen2shu4} state
  \ii[变量]{bian4liang4} variable
  \ii[循环]{xun2huan2} cyclic
  \ii[归法]{gui1fa3} induction (数学归纳法)
  \ii[广义]{guang3yi4} generalized
  \ii[术语]{shu4yu3} terms
  \ii[阶]{jie1} order
  \ii[秩]{zhi4} rank
  \ii[度]{du4} degree
  \ii[正规]{zheng4gui1} normal
  \ii[正交]{zheng4jiao1} orthogonal
\end{description}

\subsection{Solutions}
\begin{description}
  \ii[不妨假设]{bu4fang2jia3she4} without loss of generality
  \ii[解]{jie3} derive, solution
  \ii[构造]{gou4zao4} construct
  \ii[反证]{fan3zheng4} proceed by contradiction
  \ii[假设]{jia3she4} assume
  \ii[矛盾]{mao2dun4} contradiction
  \ii[只需证明]{zhi3xu1zheng4ming2} suffices to prove
  \ii[注意]{zhu4yi4} note that
  \ii[定理]{ding4li3} theorem
  \ii[引理]{yin3li3} lemma
  \ii[命题]{ming4ti2} proposition
  \ii[推论]{tui1lun4} corollary
  \ii[证明]{zheng4ming2} proof
  \ii[注]{zhu4} note / remark (as a heading)
  \ii[情况]{qing2kuang4} case
  \ii[特别地]{te4bie2de5} in particular
  \ii[述]{shu4} statement (often 上述).
  \ii[同理]{tong2li3} similarly
  \ii[显然]{xian3ran2} obviously
  \ii[得证]{de2zheng4} QED
\end{description}

\subsection{Numbers}
\begin{description}
  \ii[符号]{fu2hao4} sign
  \ii[最大值]{zui4da4zhi2} maximum
  \ii[最小值]{zui4xiao3zhi2} minimum
  \ii[数值]{shu4zhi2} number, value
  \ii[总和]{zong3he2} sum
  \ii[乘积]{cheng2ji1} product
  \ii[数线]{shu4xian4} number line
  \ii[有序对]{you3xu4dui4} ordered pair
  \ii[有序三元]{you3xu4san1yuan2} ordered triple
\end{description}

\subsection{Olympiads}
\begin{description}
  \ii[奥林匹克]{ao4lin2pi3ke4} olympiad
  \ii[国际]{guo2ji4} international
  \ii[炸]{zha4} bash
  \ii[国手]{guo2shou3} national champion; often IMO team member
  \ii[候补]{hou4bu3} alternate
  \ii[队选拔考试]{dui4xuan3ba2kao3shi4} Team Selection Test
  \ii[金牌]{jin1pai2} gold medal
  \ii[银牌]{yin2pai2} silver medal
  \ii[铜牌]{tong2pai2} bronze medal
  \ii[荣誉奖]{rong2yu4jiang3} honorable mention
\end{description}

\section{Algebra}
\begin{description}
  \ii[代数学]{dai4shu4xue2} Algebra
  \ii[实数]{shi2shu4} real number ($\mathbb R$)
  \ii[複数]{fu4shu4} complex number
  \ii[虚数]{xu1shu4} imaginary number
  \ii[实部]{shi2bu4} real part
  \ii[虚部]{xu1bu4} imaginary part
  \ii[绝对值]{jue2dui4zhi2} absolute value
  \ii[比值]{bi3zhi2} ratio
  \ii[共轭]{gong4'e4} conjugate (complex 複共轭)
  \ii[常数]{chang2shu4} constant
  \ii[线性]{xian4xing4} linear
  \ii[幂]{mi4} power
  \ii[区间]{qu1jian1} interval
  \ii[开]{kai1} open
  \ii[闭]{bi4} closed
  \ii[单位]{dan1wei4} unit
  \ii[三角函数]{san1jiao3han2shu4} trig function
  \ii[正弦]{zheng4xian2} sine
  \ii[馀弦]{yu2xian2} cosine
  \ii[微分]{wei1fen1} differentiate
  \ii[积分]{ji1fen1} integrate
\end{description}

\subsection{Algebraic Structures}
\begin{description}
  \ii[群]{qun2} group (algebraic structure)
  \ii[环]{huan2} ring
  \ii[域]{yu4} field
  \ii[运算]{yun4suan4} operations
  \ii[交换律]{jiao1huan4lv4} commutative property
  \ii[分配律]{fen1pei4lv4} distributive property
  \ii[传递律]{chuan2di4lv4} transitive property
\end{description}

\subsection{Inequalities}
\begin{description}
  \ii[不等式]{bu4deng3shi4} inequality
  \ii[界]{jie4} bound
  \ii[等号成立]{deng3hao4cheng2li4} with equality
  \ii[等号]{deng3hao4} equals sign
  \ii[等式]{deng3shi4} equality
  \ii[均值不等式]{jun1zhi2bu4deng3shi4} AM-GM Inequality
  \ii[排列不等式]{pai2lie4bu4deng3shi4} Rearrangement Inequality
  \ii[盖]{gai4} cover / majorize
\end{description}

\subsection{Functions}
\begin{description}
  \ii[函数]{han2shu4} function
  \ii[映到]{ying4dao4} project to
  \ii[单射]{dan1she4} injective
  \ii[满射]{man3she4} surjective
  \ii[一对一]{yi1dui4yi1} one-to-one
  \ii[映成]{ying4cheng2} onto
  \ii[双射]{shuang1she4} bijection
  \ii[递增]{di4zeng1} increasing
  \ii[递减]{di4jian3} decreasing
  \ii[拐点]{guai3dian3} inflection point
\end{description}

\subsection{Polynomials}
\begin{description}
  \ii[多项式]{duo1xiang4shi4} polynomial
  \ii[係数]{xi4shu4} coefficients
  \ii[首项係数]{shou3xiang4xi4shu4} leading coefficient
  \ii[次数]{ci4shu4} degree
  \ii[齐次]{qi2ci4} homogeneous
  \ii[零点]{ling2dian3} root of polynomial
  \ii[根]{gen1} root of polynomial
  \ii[二项式]{er4xiang4shi4} binomial
\end{description}

\subsection{Sequences}
\begin{description}
  \ii[数列]{shu4lie4} sequence
  \ii[递迴]{di4gui1} recursive
  \ii[係式]{xi4shi4} relation
  \ii[週期]{zhou1qi1} periodic
  \ii[算数平均数值]{suan4shu4ping2jun1shu4zhi2} arithmetic mean
  \ii[几何平均数值]{ji3he2ping2jun1shu4zhi2} geometric mean
  \ii[等差数列]{deng3cha1ji2shu4} arithmetic seq
  \ii[等比数列]{deng3bi3ji2shu4} geometric seq
\end{description}

\subsection{Linear Algebra}
\begin{description}
  \ii[线性代数]{xian4xing1dai4shu4} linear algebra
  \ii[基]{ji1} basis
  \ii[基向量]{ji1xiang1liang4} basis vector
  \ii[向量空间]{xiang1liang4kong1jian} vector space
\end{description}

\section{Combinatorics}
\begin{description}
  \ii[组合数学]{zu3he2shu4xue2} Combinatorics
  \ii[邻]{lin2} adjacent
  \ii[组]{zu3} group
  \ii[对]{dui4} pair
  \ii[将]{jiang1} now
  \ii[状态]{zhuang4tai4} state, status
  \ii[调整]{tiao2zheng3} adjust
  \ii[置换]{zhi4huan4} permutation
  \ii[令牌]{ling4pai2} token
  \ii[筹码]{chou2ma3} counter
  \ii[弹珠]{dan1zhu1} marble (toy)
  \ii[初]{chu1} initially
  \ii[逆时针]{ni4shi2zhen1} counterclockwise
  \ii[顺时针]{shun4shi2zhen1} counterclockwise
\end{description}
\subsection{Probability}
\begin{description}
  \ii[事件]{shi4jian4} event
  \ii[独立]{du2li4} independent
  \ii[机率]{ji1lv4} probability
  \ii[机率方法]{ji1lv4fang1fa3} probabilistic method
  \ii[期望值]{qi1wang4zhi2} expected value
  \ii[铜板]{tong2ban3} token
  \ii[硬币]{ying4bi4} coin
  \ii[骰子]{tou2zi5} dice
  \ii[公平]{gong1ping2} fair
  \ii[随机]{sui2ji1} random
\end{description}
\subsection{Sets}
\begin{description}
  \ii[集合]{ji2he2} set
  \ii[子集]{zi3ji2} subset
  \ii[真子集]{zhen1zi3ji2} proper subset
  \ii[族]{zu2} family (of sets)
  \ii[联集]{lian2ji2} union (of sets)
  \ii[并集]{bing4ji2} union (of sets)
  \ii[交集]{jiao1ji2} intersection (of sets)
  \ii[包含]{bao1han2} contain
  \ii[元素]{yuan2su4} element
  \ii[元素个数]{yuan2su4ge4shu4} number of elements
  \ii[互斥]{hu4zhi4} mutually exclusive (disjoint)
  \ii[空集]{kong1ji2} empty set
  \ii[非空]{fei1kong1} nonempty
  \ii[多种集]{duo1zhong3ji2} multiset
  \ii[集合划分]{ji2he2hua4fen1} partition (of set)
\end{description}
\subsection{Chess, Grids, and Labels}
\begin{description}
  \ii[棋盘]{qi2pan2} chessboard
  \ii[格]{ge2} grid
  \ii[表格]{biao1ge2} table (as a grid)
  \ii[方格]{fang1ge2} square (of a grid)
  \ii[标]{biao1} mark
  \ii[标籤]{biao1qian1} label (noun)
  \ii[编号]{bian1hao4} numbered
  \ii[兵]{bing1} pawn
  \ii[卒]{zu2} pawn
  \ii[马]{ma3} knight
  \ii[象]{xiang4} bishop
  \ii[车]{che1} rook
  \ii[后]{hou4} queen
  \ii[王]{wang2} king
\end{description}
\subsection{Game Theory}
\begin{description}
  \ii[甲]{jia3} A (Alice)
  \ii[乙]{yi3} B (Bob)
  \ii[丙]{bing3} C
  \ii[丁]{ding1} D
  \ii[轮流]{lun2liu2} take turns
  \ii[回合]{hui2he2} round
  \ii[移动]{yi2dong4} move
  \ii[操作]{cao1zuo4} operation
  \ii[选择]{xuan3ze2} select
  \ii[移走]{yi2zou3} remove
  \ii[胜]{sheng4} victory
  \ii[获胜]{huo4sheng4} to win
  \ii[输]{shu1} lose
  \ii[必胜法]{bi4sheng4fa3} winning strategy
  \ii[策略]{ce4lue4} tactic
\end{description}

\subsection{Graph Theory}
\begin{description}
  \ii[图论]{tu2lun4} graph theory
  \ii[图]{tu2} graph
  \ii[同构]{tong2gou4} isomorphic
  \ii[自同构]{zi4tong2gou4} automorphism
  \ii[点色数]{dian3se4shu4} chromatic number
  \ii[入度]{ru4du4} indegree
  \ii[出度]{chu1du4} outdegree
  \ii[子图]{zi3tu2} subgraph
  \ii[自环]{zi4huan2} cycle
  \ii[路径]{lu4jing4} path
  \ii[简单图]{jian3dan1tu2} simple graph
  \ii[完全图]{wan2quan2tu2} complete graph
  \ii[二部图]{er4bu4tu2} bipartite graph
  \ii[$n$-部图]{n-bu4tu2} $n$-partite graph
  \ii[有向图]{you3xiang4tu2} directed graph
  \ii[无环图]{wu2huan2tu2} acyclic graph
  \ii[补图]{bu3tu2} complement of graph
  \ii[树]{shu4} tree
  \ii[连通图]{lian2tong1tu2} connected graph
  \ii[连通分量]{lian2tong1fen1liang4} connected component
  \ii[强连通]{qiang2lian2tong1} strongly connected
  \ii[正则图]{zheng4ze2tu2} regular graph
\end{description}

\subsection{Strings}
\begin{description}
  \ii[字符串]{zi4fu2chuan4} string / word
  \ii[回文]{hui2wen2} palindrome
  \ii[连接]{lian2jie1} concatenate
  \ii[空字符串]{kong1zi4fu2chuan4} empty string
  \ii[位数]{wei4shu4} digit
  \ii[附加]{fu4jia1} append
  \ii[十进制数]{shi2jin4zhi4shu4} decimal digit
  \ii[底数]{di1shu4} base / radix
  \ii[二进制]{er4jin4zhi4} binary
\end{description}

\subsection{Colors}
\begin{description}
  \ii[染色]{ran3se4} coloring
  \ii[红色]{hong2se4} red
  \ii[绿色]{lv4se4} green
  \ii[蓝色]{lan2se4} blue
  \ii[黄色]{huang2se4} yellow
  \ii[白]{bai2} white
  \ii[黑]{hei1} black
\end{description}

\section{Geometry}
\begin{description}
  \ii[几何学]{ji3he2xue2} Geometry
  \ii[纯几]{chun2ji3} synthetic
  \ii[中点]{zhong1dian3} midpoint
  \ii[对分]{dui4fen1} bisect
  \ii[共线]{gong4xian4} collinear
  \ii[共圆]{gong4yuan2} concyclic (``are concyclic'')
  \ii[共点]{gong4dian3} concurrent
  \ii[圆内接]{yuan2nei4jie1} cyclic (``cyclic quadrilateral'')
  \ii[角]{jiao3} angle
  \ii[直角]{zhi2jiao3} right angle
  \ii[全等的]{quan2deng3de5} congruent
  \ii[内接于]{nei4jie1yu2} is inscribed in
  \ii[介于]{jie4yu2} between
  \ii[阵点]{zhen4dian3} lattice point
  \ii[交于]{jiao1yu2} intersect at (verb)
  \ii[交点]{jiao1dian3} intersect point
  \ii[定点]{ding4dian3} fixed point
  \ii[相似]{xiang1si4} similar (esp. 相似三角形)
  \ii[位似]{wei4si4} homothety
  \ii[镜射]{jing4she4} reflection
  \ii[对称]{dui4chen4} symmetric
  \ii[轨迹]{gui3ji4} locus
\end{description}

\subsection{Lines}
\begin{description}
  \ii[垂直]{chui2zhi2} perpendicular
  \ii[垂直平分线]{chui2zhi2ping2fen1xian4} perpendicular bisector
  \ii[垂足]{chui2zu2} foot
  \ii[平行]{ping2xing2} parallel
  \ii[距离]{ju4li2} distance
  \ii[长度]{chang2du4} length
  \ii[延长]{yan2chang2} extend
  \ii[落]{luo4} lies on
  \ii[线]{xian4} line
  \ii[折线]{zhe2xian4} broken line
  \ii[封闭]{feng1bi4} closed
  \ii[射线]{she4xian4} ray
  \ii[线段]{xian4duan4} segment
\end{description}

\subsection{Polygons}
\begin{description}
  \ii[多边形]{duo1bian1xing2} polygon
  \ii[简单]{jian3dan1} simple (polygon)
  \ii[四边形]{si4bian1xing2} quadrilateral
  \ii[五边形]{wu3bian1xing2} pentagon
  \ii[正多边形]{zheng4duo1bian1xing2} regular polygon
  \ii[凸]{tu1} convex
  \ii[凹]{ao1} concave
  \ii[形]{xing2} shape
  \ii[界]{jie4} boundary
  \ii[内部]{nei4bu4} interior
  \ii[界分]{jie4fen1} boundary points
  \ii[週长]{zhou1chang2} perimeter
  \ii[面积]{mian4ji1} area
  \ii[对角线]{dui4jiao3xian4} diagonal
  \ii[凸包]{tu1bao1} convex hull
  \ii[边]{bian1} side, edge
  \ii[边长]{bian1chang2} side length
  \ii[区域]{qu1yv4} region
\end{description}

\subsection{Polyhedrons}
\begin{description}
  \ii[多面体]{duo1mian4ti3} polyhedron
  \ii[面]{mian4} face
  \ii[立方体]{li4fang1ti3} cube
  \ii[锥体]{zhui1ti3} pyramid
  \ii[四面体]{si4mian4ti3} tetrahedron
  \ii[圆锥]{yuan2zhui1} cone
  \ii[球面]{qiu2mian4} (surface of) sphere
  \ii[球]{qiu2} closed ball
  \ii[体积]{ti3ji1} volume
\end{description}

\subsection{Coordinates}
\begin{description}
  \ii[坐标]{zuo4biao1} coordinates
  \ii[笛卡尔平面]{di2ka3'er3ping2mian4} Cartesian (coordinate) plane
  \ii[笛卡尔空间]{di2ka3'er3kong1jian1} Cartesian space
  \ii[平面]{ping2mian4} plane
  \ii[超平面]{chao1ping2mian4} hyperplane
  \ii[原点]{yuan2dian3} origin
  \ii[轴]{zhou2} axis
  \ii[抛物线]{pao1wu4xian4} parabola
\end{description}

\subsection{Quadrilaterals}
\begin{description}
  \ii[菱形]{ling2xing2} rhombus
  \ii[梯形]{ti1xing2} trapezoid
  \ii[平行四边形]{ping2xing2si4bian1xing2} parallelogram
  \ii[长方形]{chang2fang1xing2} rectangle
  \ii[矩形]{ju3xing2} rectangle
  \ii[正方形]{zheng4fang1xing2} square
  \ii[单位正方形]{dan1wei4zheng4fang1xing2} unit square
\end{description}

\subsection{Circles}
\begin{description}
  \ii[圆]{yuan2} circle
  \ii[圆周]{yuan2zhou1} circumference
  \ii[圆弧]{yuan2hu2} arc
  \ii[单位圆]{dan1wei4yuan2} unit circle
  \ii[弧]{hu2} arc
  \ii[弦]{xian2} chord
  \ii[半径]{ban4jing4} radius
  \ii[直径]{zhi2jing4} diameter
  \ii[圆心]{yuan2xin1} center
  \ii[相切]{xiang1qie1} tangent
  \ii[内切]{nei4qie1} internally tangent
  \ii[切点]{qie1dian3} tangency point
  \ii[圆幂]{yuan2mi4} power of a point
  \ii[根轴]{gen1zhou2} radical axis
  \ii[根心]{gen1xin1} radical center
\end{description}

\subsection{Types of Triangles}
\begin{description}
  \ii[三角形]{san1jiao3xing2} triangle
  \ii[等腰]{deng3yao1} isosceles
  \ii[等边]{deng3bian1} equilateral
  \ii[等角]{deng3jiao3} equilangular
  \ii[等边三角形]{deng3bian1san1jiao3xing2} equilateral triangle
  \ii[直角三角形]{zhi2jiao3san1jiao3xing2} right triangle
  \ii[锐角]{rui4jiao3} acute
  \ii[钝角]{dun4jiao3} obtuse
  \ii[不等边三角形]{bu4deng3bian1san1jiao3xing2} scalene triangle
\end{description}

\subsection{Centers of a Triangle}
\begin{description}
  \ii[顶点]{ding3dian3} vertex
  \ii[重心]{zhong4xin1} centroid/barycenter
  \ii[垂心]{chui2xin1} orthocenter
  \ii[内心]{nei4xin1} incenter
  \ii[内接圆]{nei4jie1yuan2} incircle
  \ii[外心]{wai4xin1} circumcenter
  \ii[外接圆]{wai4jie1yuan2} circumcircle
  \ii[旁心]{pang2xin1} excenter
  \ii[傍心]{bang4xin1} excenter
  \ii[旁切圆]{pang2qie1yuan2} excircle
  \ii[伪内切圆]{wei3nei4qie1yuan2} mixtilinear incircle
  \ii[高]{gao1} altitude / height
  \ii[底]{di3} base
  \ii[中线]{zhong1xian4} median
  \ii[角平分线]{jiao3ping2fen1xian4} angle bisector
  \ii[陪位中线]{pei2wei4zhong1xian4} symmedian
  \ii[等角中线]{deng3jiao3zhong1xian4} symmedian
  \ii[陪位重心]{pei2wei4zhong4xin1} symmedian point
  \ii[等角共轭]{deng3jiao3gong4'e4} isogonal conjugate
  \ii[等距共轭]{deng3ju4gong4'e4} isotomic conjugate
  \ii[热尔岗点]{re4'er3gang3dian3} Gergonne point
  \ii[纳格尔点]{na4ge2'er3dian3} Nagel point
\end{description}

\subsection{Higher Geometry}
\begin{description}
  \ii[反演]{fan3yan3} inversion
  \ii[射影]{she4ying3} projective
  \ii[交比]{jiao1bi3} cross-ratio
  \ii[调和]{tiao2he2} harmonic
  \ii[调和点列]{tiao2he2dian3lie4} harmonic bundle
  \ii[行列式]{hang2lie4shi4} determinant
  \ii[极]{ji2} pole
  \ii[极线]{ji2xian4} polar
  \ii[完全四边形]{wan2quan2si4bian1xing2} complete quadrilateral
  \ii[射影变换]{she4ying3bian4huan4} projective transformation
  \ii[圆锥曲线]{yuan2zhui1qu1xian4} conic section
  \ii[无穷远点]{wu2qiong2yuan3dian3} point at infinity
  \ii[向量]{xiang4liang4} vector
\end{description}

\section{Number Theory}
\begin{description}
  \ii[数论]{shu4lun4} Number Theory
  \ii[整数]{zheng3shu4} integer
  \ii[正数]{zheng4shu4} positive number
  \ii[负数]{fu4shu4} negative number
  \ii[自然数]{zi4ran2shu4} natural number
  \ii[非负]{fei1fu4} nonnegative
  \ii[偶数]{ou3shu4} even number
  \ii[奇数]{ji1shu4} odd number
  \ii[分数]{fen1shu4} fraction
  \ii[分子]{fen1zi3} numerator
  \ii[分母]{fen1mu3} denominator
  \ii[有理数]{you3li3shu4} rational number
  \ii[无理数]{wu2li3shu4} irrational number
  \ii[质数]{zhi2shu4} prime number
  \ii[素数]{su4shu4} prime number
  \ii[倍数]{bei4shu4} multiple
  \ii[因数]{yin1shu4} divisor/factor
  \ii[整除]{zheng3chu2} divides
  \ii[互质]{hu4zhi4} relatively prime
  \ii[合数]{he2shu4} composite number
  \ii[平方数]{ping2fang1shu4} perfect square
  \ii[立方数]{li4fang1shu4} perfect cube
  \ii[次方]{ci4fang1} power
  \ii[模]{mo2} mod
  \ii[原根]{yuan2gen1} primitive root
  \ii[剩馀]{sheng4yu2} remainder
  \ii[二次剩馀]{er4ci4sheng4yu2} quadratic residue
  \ii[阶乘]{jie1cheng2} factorial
\end{description}

\section{Calculus}
\begin{description}
  \ii[微积分]{wei2ji1fen1} calculus
  \ii[极限]{ji2xian4} limit
  \ii[微分]{wei2fen1} derivative
  \ii[积分]{ji1fen1} integral
\end{description}

\section{Miscellaneous}
\subsection{Category Theory}
\begin{description}
  \ii[范畴论]{fan4chou2lun4} category theory
  \ii[范畴]{fan4chou2} category
  \ii[核]{he2} kernel
  \ii[预层]{yu4ceng2} presheaf
  \ii[函子]{han2zi3} functor
  \ii[自然变换]{zi4ran2bian4huan4} natural transformation
  \ii[类]{lei4} class
  \ii[物件]{wu4jian4} objects
  \ii[态射]{tai4she4} morphism
  \ii[箭号]{jian4hao4} arrows
  \ii[二元运算]{er4yuan2yun4suan4} binary operation
  \ii[结合律]{jie2he2lv4} associative law
  \ii[複合]{fu4he2} composition
\end{description}

\subsection{Analysis}
\begin{description}
  \ii[解析]{jie3xi1} analysis, analytic
\end{description}

\subsection{Flavor text}
\begin{description}
  \ii[捲饼]{juan3bing3} burrito
  \ii[鸽子]{ge1zi5} pigeon
  \ii[河]{he2} river
  \ii[睡莲科]{shui4lian2ke1} lilypad
  \ii[药丸]{yao4wan2} pill
  \ii[体重]{ti3zhong4} weight
  \ii[冰块]{bing1kuai4} ice cube
  \ii[青蛙]{qing1wa1} frog
  \ii[罐]{guan4} jar
  \ii[石头]{shi2tou2} rock
  \ii[桶]{tong3} bucket
  \ii[蟾蜍]{chan2chu2} toad
  \ii[朋友]{peng2you3} friend
  \ii[杨辉三角]{yang1hui4san1jiao3} Pascal triangle
  \ii[忒修斯]{te4xui1si1} Theseus
  \ii[网球]{wang3qiu2} tennis
  \ii[循环赛]{xun2huan2sai4} round-robin tournament
\end{description}

\end{document}
